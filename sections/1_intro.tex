\section{Introduction}

Camera calibration, also known as camera resectioning, is the process of determining the intrinsic and extrinsic parameters of a camera. The intrinsic parameters deal with the camera's internal characteristics, while the extrinsic parameters describe its position and orientation in the world. The knowledge of the accurate values of these values parameters are essential, as it enables us to create a mathematical model which describes how a camera projects 3D points from a scene onto the 2D image it captures. The importance of a well-calibrated camera becomes very apparent in photogrammetric applications, where precise measurements of 3-dimensional physical objects are derived from photographic images.

Photogrammetry is the science of obtaining accurate measurements of 3-dimensional physical objects through photographic imagery. Photogrammetry was first employed by Prussian architect Albrecht Meydenbauer in the 1860s, who used photogrammetric techniques to create some of the most detailed topographic plans and elevations drawings \footcite[][1]{albertzLookBack2007}. Today, photogrammetric techniques are used in a multitude of applications spanning diverse fields, including but not limited to: 3D-model generation, computer vision, topographical mapping, medical imaging, and forensic analysis. 

While camera calibration is essential in ensuring the accuracy of photogrammetric applications, it itself also relies on these very same photogrammetric techniques in order to estimate these parameters. In essence, the developments of photogrammetry and camera calibration are closely intertwined, underscoring the essential relationship between photogrammetry and camera calibration.

\subsection{Problem Statement}

While manufacturers of cameras often report parameters of cameras, such as the nominal focal length and pixel sizes of their camera sensor, these figures are typically approximations which can vary from camera to camera, particularly in consumer-grade cameras. As such, the use of these estimates by manufacturers are unsuitable in developing camera models for applications requiring high accuracy. Combined with the potential for manufacturing defects as well as unknown lens distortion coefficients further necessitates the need for a reliable method for determining the parameters of a camera. 

Camera calibration emerges as the answer to these problems, allowing us to create very accurate models for the camera as well as generate estimates for its parameters. As such, it is important that we understand the various techniques and strategies use in camera calibration, and how they can be applied to ensure high accuracy.
