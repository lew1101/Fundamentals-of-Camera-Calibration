\section{Introduction}

3D reconstruction from multiple images is the process of generating an accurate 3D model of a physical object from its 2D images. The process employs techniques from the broader science of photogrammetry, which is concerned with obtaining precise measurements of 3-dimensional physical objects from photographic images. The use of photogrammetry as a method of 3D reconstruction was first employed by Prussian architect Albrecht Meydenbauer in the 1860s, who created some of the earliest topographic plans and elevations drawings realized using photogrammetric techniques \autocite{ices2017}. Today, photogrammetry has applications in a wide variety of fields, including but not limited to: architecture, engineering, forensics, medicine, and geology.

\subsection{Overview of 3D Reconstruction}



The task of generating an accurate 3D model from various 2D images is a multi-step process, and consists of:




and they mainly fall into two main categories. The first method is based on the cross-referencing of keypoints between the images. Computer algorithms such as SIFT (Scale Invariant Feature Transform) and SURF (Speeded-Up Robust Features) 

projective geometry to triangulate the position of keypoints of the objects. These keypoints are cross-referenced between the images either by hand or using a computer algorithm, and then 


Camera calibration is a important 







