\section{Extracting Parameters} \label{sec:extract}

Once we have solved for the projection for the projection matrix $P$, we can then decompose it and extract the intrinsic and extrinsic parameters. Recall that $P = K\,[\,R\;|\;t\,]$. Distributing $K$ into $[\,R\;|\;t\,]$ yields:
\begin{align} \label{eq:param}
    P & = [\,KR\;|\;Kt\,]     \nonumber                     \\
      & = [\,KR\;|\;-KRc_w\,]  \nonumber                    \\
      & = [\,Q\;|\;-Qc_w\,]      \qquad \text{with } Q = KR
\end{align}
Analyzing the above equation reveals that $Q$ corresponds to the first three columns of the projection matrix. i.e.
\begin{equation*}
    Q \equiv
    \begin{bmatrix}
        p_{11} & p_{12} & p_{13} \\
        p_{21} & p_{22} & p_{23} \\
        p_{31} & p_{32} & p_{33}
    \end{bmatrix}
    =
    \underbrace{
        \begin{bmatrix}
            f_x & 0   & c_x \\
            0   & f_y & c_y \\
            0   & 0   & 1
        \end{bmatrix}
    }_{\mathlarger{K}}
    \underbrace{
        \begin{bmatrix}
            r_{11} & r_{12} & r_{13} \\
            r_{21} & r_{22} & r_{23} \\
            r_{31} & r_{32} & r_{33} \\
        \end{bmatrix}
    }_{\mathlarger{R}}
\end{equation*}

\subsection{RQ Decomposition} \label{subsec:rq}
Since $K$ is in the form of an \emph{upper right triangular matrix} and $R$ is an \emph{orthonormal matrix}, it is possible to decompose $P$ and find unique solutions for $K$ and $R$ using a method called \textbf{RQ decomposition}. There are various algorithms that exist to do so, such as the \emph{Graham-Schmidt Process} and the \emph{Householder transformation} method, however, the minute details of how they work fall beyond the scope of this paper. Instead, we use \texttt{SciPy}, a \emph{Python} library, to perform the decomposition.

\subsection{Extracting the Camera Position}

From equation \ref{eq:param}, we see that $-Qc_w$ is equal to the last column of $P$. Thus:
\begin{gather}
    \begin{bmatrix}
        p_{14} \\ p_{24} \\ p_{34}
    \end{bmatrix} = -Qc_w   \nonumber \\
    \Rightarrow \boxed{c_w = -Q^{-1}
    \begin{bmatrix}
        p_{14} \\ p_{24} \\ p_{34}
    \end{bmatrix}}
\end{gather}
where $Q^{-1}$ denotes the inverse of $Q$.

\subsection{Extracting Orientation as Angles}

When constructing the extrinsic matrix in section \ref{sec:extrinsics}, we defined the rotation of the camera as a $3 \times 3$ matrix, where the rows represented the unit vectors $\hat{x}_c$, $\hat{y}_c$, and $\hat{z}_c$ after the rotation is performed. However, it is easier to interpret the rotations in terms of angles because angles provide a more intuitive and human-understandable representation of orientation. Specifically, we can employ \textbf{Tait-Bryan angles}, where the rotation is decomposed into three sequential rotations about each of the principle axes. These angles offer a straightforward way to understand how an object is rotated in terms of pitch, yaw, and roll. We employ \emph{x-y-z} Tait-Bryan angles, a variant where the $x$ rotation is performed first, then the $y$ rotation, and finally the $z$ rotation. This can be mathematically represented thus:
\begin{equation}
    R \equiv R_z(\gamma)R_y(\beta)R_x(\alpha)
\end{equation}
where:
\begin{subequations}
    \begin{gather}
        R_x(\alpha) =
        \begin{bmatrix}
            1 & 0            & 0             \\
            0 & \cos(\alpha) & -\sin(\alpha) \\
            0 & \sin(\alpha) & \cos(\alpha)
        \end{bmatrix} \\
        R_y(\beta) =
        \begin{bmatrix}
            \cos(\beta)  & 0 & -\sin(\beta) \\
            0            & 1 & 0            \\
            -\sin(\beta) & 0 & -\cos(\beta)
        \end{bmatrix} \\
        R_z(\gamma) =
        \begin{bmatrix}
            \cos(\gamma) & -\sin(\gamma) & 0 \\
            \sin(\gamma) & \cos(\gamma)  & 0 \\
            0            & 0             & 1
        \end{bmatrix}
    \end{gather}
\end{subequations}
We can then multiply the matrices together to obtain a single $3 \times 3$ matrix:
\begin{equation} \label{eq:rot}
    R = \scalemath{0.85}{
        \begin{bmatrix}
            \cos(\beta)\,\cos(\gamma) & \sin(\alpha)\,\sin(\beta)\,\cos(\gamma) - \cos(\alpha)\,\sin(\gamma) & \cos(\alpha)\,\sin(\beta)\,\cos(\gamma) + \sin(\alpha)\,\cos(\gamma) \\
            \cos(\beta)\,\sin(\gamma) & \sin(\alpha)\,\sin(\beta)\,\sin(\gamma) + \cos(\alpha)\,\cos(\gamma) & \cos(\alpha)\,\sin(\beta)\,\sin(\gamma) - \sin(\alpha)\,\cos(\gamma) \\
            -\sin(\beta)              & \sin(\alpha)\,\cos(\beta)                                            & \cos(\alpha)\,\cos(\beta)
        \end{bmatrix}
    }
\end{equation}
Since we have already determined for the parameters of the rotation matrix, we can solve for the angles $\alpha$, $\beta$, and $\gamma$ by deriving formulas for them using equation \ref{eq:rot}:
\begin{gather}
    r_{31} = -\sin(\beta) \nonumber \\
    \Rightarrow \boxed{\beta = \sin^{-1}(-r_{31})}
\end{gather}
\begin{gather}
    r_{32} = \sin(\alpha)\cos(\beta) \nonumber \\
    \Rightarrow \sin(\alpha)  = \frac{r_{32}}{\cos(\beta)} = \frac{r_{32}}{\cos(\sin^{-1}(-r_{31}))} \nonumber \\
    \Rightarrow \boxed{\alpha = \sin^{-1}\left(\frac{r_{32}}{\sqrt{1-r_{31}^2}}\right)}
\end{gather}
\begin{gather}
    r_{21} = \cos(\beta)\sin(\gamma) \nonumber                                                                 \\
    \Rightarrow \sin(\gamma)  =\frac{r_{21}}{\cos(\beta)}  = \frac{r_{21}}{\cos(\sin^{-1}(-r_{31}))} \nonumber \\
    \Rightarrow \boxed{\gamma= \sin^{-1}\left(\frac{r_{21}}{\sqrt{1-r_{31}^2}}\right)}
\end{gather}
