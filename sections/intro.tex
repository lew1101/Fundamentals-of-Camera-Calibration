\section{Introduction}

Camera calibration is an important process in computer vision and computer graphics which involves determining the parameters of a camera. The knowledge of these parameters are essential, because it allows us to create a mathematical model which accurately describes the camera. Without a well-calibrated camera, images captured may suffer from inaccuracies and distortion, making calibration an indispensable step in a wide array of applications.

\subsection{Problem Statement}

While manufacturers of cameras often report parameters of cameras, such as the nominal focal length and pixel sizes of their camera sensor, these figures are typically approximations which can vary from camera to camera, particularly in consumer-grade cameras. As such, the use of these estimates by manufacturers are unsuitable in developing camera models for applications requiring high accuracy. Combined with the potential for manufacturing defects as well as unknown lens distortion coefficients further necessitates the need for a reliable method for determining the parameters of a camera. 

Camera calibration emerges as the answer to these problems, allowing us to create very accurate estimates for the parameters of a camera. As such, it is important that we ac

Photogrammetry, as a comprehensive science, concerns itself with obtaining precise measurements of 3-dimensional physical objects from photographic images. 

It was first employed by Prussian architect Albrecht Meydenbauer in the 1860s, who used photogrammetric techniques to create some of the most detailed topographic plans and elevations drawings \footcite{albertzLookBack2007}. 
Camera calibration borrows techniques from photogrammetry  

Today, photogrammetric techniques are used in a multitude of applications spanning diverse fields, including computer vision, topographical mapping, medical imaging, and forensic analysis. 



Importance of reserach question

In order to accurately determine the position of 3D points based on data from multiple 2D images, we must have knowledge of the parameters of the camera.

This process of calculating








