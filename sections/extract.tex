\section{Extracting Parameters}

Once we have solved for the projection for the projection matrix $P$, we can then extract the intrinsic and extrinsic parameters. We know that
\begin{equation*}
    P =
    \begin{bmatrix}
        p_{11} & p_{12} & p_{13} & p_{14} \\
        p_{21} & p_{22} & p_{23} & p_{24} \\
        p_{31} & p_{32} & p_{33} & p_{34}
    \end{bmatrix}
    =
    \underbrace{
        \begin{bmatrix}
            f_x & 0   & c_x & 0 \\
            0   & f_y & c_y & 0 \\
            0   & 0   & 1   & 0
        \end{bmatrix}
    }_{\mathlarger{M_{int}}}
    \underbrace{
        \begin{bmatrix}
            r_{11} & r_{12} & r_{13} & t_x \\
            r_{21} & r_{22} & r_{23} & t_y \\
            r_{31} & r_{32} & r_{33} & t_z \\
            0      & 0      & 0      & 1
        \end{bmatrix}
    }_{\mathlarger{M_{ext}}} \tag{\ref{eq:projection} revisited}
\end{equation*}

\begin{equation}
    \begin{bmatrix}
        p_{11} & p_{12} & p_{13} \\
        p_{21} & p_{22} & p_{23} \\
        p_{31} & p_{32} & p_{33}
    \end{bmatrix}
    =
    \begin{bmatrix}
        f_x & 0   & c_x \\
        0   & f_y & c_y \\
        0   & 0   & 1
    \end{bmatrix}
    \begin{bmatrix}
        r_{11} & r_{12} & r_{13} \\
        r_{21} & r_{22} & r_{23} \\
        r_{31} & r_{32} & r_{33}
    \end{bmatrix}
    =KR
\end{equation}

Since $K$ is in the form of an \emph{upper right triangular matrix} and $R$ is an \emph{orthonormal matrix}, we can find unique solutions for $K$ and $R$ using a method called \emph{RQ decomposition}.

\subsection{RQ Decomposition}

RQ decomposition is a technique which allows us to uniquely decompose a matrix $A$ into a product $A=RQ$,

Since

\subsection{Extracting the Translation Vector}

\begin{equation}
    \begin{bmatrix}
        p_{14} \\ p_{24} \\ p_{34}
    \end{bmatrix}
    =
    \begin{bmatrix}
        f_x & 0   & c_x \\
        0   & f_y & c_y \\
        0   & 0   & 1
    \end{bmatrix}
    \begin{bmatrix}
        t_x \\ t_y \\ t_z
    \end{bmatrix}
    =K\vec{t}
\end{equation}

\subsection{Extracting Orientation as Angles}

When constructing the extrinsic matrix in section \ref{sec:extrinsics}, we defined the rotation matrix as the 

We can represent the rotation in terms of \emph{Tait-Bryan Angles}

\begin{equation}
    R \equiv R_z(\gamma)R_y(\beta)R_x(\alpha)
\end{equation}

\begin{subequations}
    \begin{gather}
        R_x(\alpha) =
        \begin{bmatrix}
            1 & 0            & 0             \\
            0 & \cos(\alpha) & -\sin(\alpha) \\
            0 & \sin(\alpha) & \cos(\alpha)
        \end{bmatrix} \\
        R_y(\beta) =
        \begin{bmatrix}
            \cos(\beta)  & 0 & -\sin(\beta) \\
            0            & 1 & 0            \\
            -\sin(\beta) & 0 & -\cos(\beta)
        \end{bmatrix} \\
        R_z(\gamma) =
        \begin{bmatrix}
            \cos(\gamma) & -\sin(\gamma) & 0 \\
            \sin(\gamma) & \cos(\gamma)  & 0 \\
            0            & 0             & 1
        \end{bmatrix}
    \end{gather}
\end{subequations}

\newcommand{\ca}{\mathrm{c}}
\newcommand{\sa}{\mathrm{s}}

\begin{align}
    R & =
    \begin{bmatrix}
        0 & \cos(\alpha) & -\sin(\alpha) \\
        0 & \sin(\alpha) & \cos(\alpha)  \\
        1 & 0            & 0             \\
    \end{bmatrix}
    \begin{bmatrix}
        \cos(\beta)  & 0 & -\sin(\beta) \\
        0            & 1 & 0            \\
        -\sin(\beta) & 0 & -\cos(\beta)
    \end{bmatrix}
    \begin{bmatrix}
        \cos(\gamma) & -\sin(\gamma) & 0 \\
        \sin(\gamma) & \cos(\gamma)  & 0 \\
        0            & 0             & 1
    \end{bmatrix} \nonumber \\
      & =
    \scalemath{0.85}{
        \begin{bmatrix}
            \cos(\beta)\,\cos(\gamma) & \sin(\alpha)\,\sin(\beta)\,\cos(\gamma) - \cos(\alpha)\,\sin(\gamma) & \cos(\alpha)\,\sin(\beta)\,\cos(\gamma) + \sin(\alpha)\,\cos(\gamma) \\
            \cos(\beta)\,\sin(\gamma) & \sin(\alpha)\,\sin(\beta)\,\sin(\gamma) + \cos(\alpha)\,\cos(\gamma) & \cos(\alpha)\,\sin(\beta)\,\sin(\gamma) - \sin(\alpha)\,\cos(\gamma) \\
            -\sin(\beta)              & \sin(\alpha)\,\cos(\beta)                                            & \cos(\alpha)\,\cos(\beta)
        \end{bmatrix}
    }
\end{align}

We have that
\begin{equation*}
    r_{31} = -\sin(\beta) \nonumber
\end{equation*}
\begin{equation}
    \Rightarrow \beta = \sin^{-1}(-r_{31})
\end{equation}

\begin{equation*}
    r_{21} = \cos(\beta)\sin(\gamma)
\end{equation*}
\begin{align}
    \begin{split}
        \Rightarrow \gamma & = \sin^{-1}\left(\frac{r_{21}}{\cos(\beta)}\right)               = \sin^{-1}\left(\frac{r_{21}}{\cos(\sin^{-1}(-r_{31}))}\right) \\
        & = \sin^{-1}\left(\frac{r_{21}}{\sqrt{1-r_{31}^2}}\right)
    \end{split}
\end{align}

\begin{equation*}
    r_{32} = \sin(\alpha)\cos(\beta)
\end{equation*}
\begin{align}
    \begin{split}
        \Rightarrow \alpha & = \sin^{-1}\left(\frac{r_{32}}{\cos(\beta)}\right)               = \sin^{-1}\left(\frac{r_{32}}{\cos(\sin^{-1}(-r_{31}))}\right) \\
        & = \sin^{-1}\left(\frac{r_{32}}{\sqrt{1-r_{31}^2}}\right)
    \end{split}
\end{align}
