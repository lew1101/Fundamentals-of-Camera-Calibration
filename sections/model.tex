\section{Approach} 

\subsection{Camera Model} \label{sec:camera_model}

A camera model is a projection model that approximates the function of a camera by describing a mathematical relationship between points in 3D space and its projection onto the sensor grid of the camera. In order to accurately model a camera, we must first understand the general workings of a camera.

Modern lens cameras are highly sophisticated, built with an array of complex mechanisms and a wide range of features. The complexity of cameras can be better understood by breaking down their components down into three main elements critical to image projection: the lens, the aperture, and the sensor grid (CCD). 

\begin{figure}[H]
    \centering
    \includegraphics[width=0.5\textwidth]{images/lens_camera}
    \caption{Lens camera. Adapted from \cite{coltonPhysics1232012}} \label{fig:lens_camera}
\end{figure}

The lenses 


However, we can simplify the model of the lens camera by collapsing the mechanisms of the camera into 3 main functional components that are important to the image projection: the lens, the aperture, and the sensor grid (CCD). This simplified model is visualized in Figure \ref{fig:lens_camera}. The lens focuses incoming light rays towards the aperture, before they project inverted onto the sensor grid. However, even this simplified model of a lens camera is too complex to model, as there is no simple mathematical equation which accurately describes the behavior of a lens. As such, we can further simplify our camera model by building upon the pinhole camera model, which is one of the simplest and most commonly used camera models in camera calibration.



A pinhole camera is a simple camera without a lens. Instead, it relies on the use of a tiny hole as the aperture of the camera, and light rays pass through the hole, projecting an inverted image onto the

\begin{figure}[H]
    \centering
    \includegraphics[width=0.7\textwidth]{images/pinhole_vs_lens}
    \caption{Difference between a pinhole camera and a lens camera. Adapted from \cite{leCameraModel2018}}
\end{figure}



There are a few assumptions which are made by the pinhole camera model:

Extremely simple model for imaging geometry
Doesn't strictly apply
Mathematically convenient acceptable approximation.
\begin{itemize}
    \item T
\end{itemize}

The pinhole camera model does not accurately describe the true workings of a camera, as some of the  effects that the model fails to account for can be compensated the errors which results from these assumptions are sufficiently small to be neglected if a high quality camera is used. Additionally,

\begin{figure}[H]
    \centering
    \includegraphics[width=0.9\textwidth]{figures/imaging_model}
    \caption{Pinhole camera model.}
\end{figure}

\subsubsection{Geometry}


For our camera model, we will establish 3 frames of


\begin{figure}[H]
    \centering
    \includegraphics[width=0.9\textwidth]{figures/coord_conversions}
    \caption{Coordinate remappings.}
\end{figure}





There are countless different approaches one could take to calibrate a camera, 



however they all build upon techniques first described in multiple highly influential papers, most notably Tsai's ``A Versatile Camera Calibration Technique for High-Accuracy 3D Machine Vision Metrology Using Off-the-shelf TV Cameras and Lenses'' and Zhang's ``A Flexible New Technique for Camera Calibration''. 

\subsection{Calibration Object}

Calibration techniques can be roughly separated into 3 categories, based on the dimension of the calibration object used \footcite{zhangCameraCalibration2007}:

