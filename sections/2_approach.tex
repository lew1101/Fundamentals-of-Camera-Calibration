\section{Approach} 

There are many techniques one could take to calibrate a camera, 




\subsection{Camera Model} \label{sec:camera_model}

A camera model is a projection model which approximates the function of a camera by describing a mathematical relationship between points in 3D space and its projection onto the sensor grid of the camera. In order to construct such a model, we must first understand the general workings of a camera.

The modern lens camera is highly sophisticated, built with an array of complex mechanisms and a wide range of features such as zoom and autofocus. However, we only need to focus on its three principal elements critical to image projection: the lens, the aperture, and the sensor grid (CCD). 

\begin{itemize}[leftmargin=!, itemindent=-5ex]
    \item \textbf{Lens} -- Focuses incoming light rays and projects it onto the sensor grid. Modern cameras have compound lenses (lenses made up of several lens elements) in order to minimize undesired effects such as aberration, blurriness, and distortion. 
    \item \textbf{Aperture} -- Controls the amount of light that reaches the sensor. By adjusting the aperture size, the exposure and depth of field can be modified.
    \item \textbf{Sensor Grid} -- Captures incoming light rays and converts this information into pixels on an image. 
\end{itemize}

\begin{figure}[H]
    \centering
    \includegraphics[width=0.5\textwidth]{images/lens_camera}
    \caption{Lens camera. Adapted from \cite{coltonPhysics1232012}} \label{fig:lens_camera}
\end{figure}

However, it is impossible to construct a model which is both simple and exact for the lens cameras, as the behavior of lenses are very complex. As such, it is mathematically convenient to approximate the camera as a pinhole camera. In doing so, we ignore lens distortion, but it distills the behavior of a camera to its most fundamental and essential dynamics: the projection of points in 3D space onto the flat 2D sensor plane. 

\subsubsection{Pinhole Camera Model}

A pinhole camera is a simple camera without a lens. It instead relies on the use of a tiny hole as the aperture of the camera, and light rays pass through the hole, projecting an inverted image onto the image plane. The pinhole camera model is based on the pinhole camera, however it goes further by making the assumptions that the aperture is infinitely small. This means that any incoming light ray can only travel straight through the pinhole, and that a point in space can only map to one single point on image plane. 

If necessary, one could reintroduce distortion and shear terms in order to minimize the error, but this is often not needed for low to medium precision applications, as the distortion of modern lenses are already minimal. As such, its ease of use has led it to become one of the most frequently employed camera models in the field of camera calibration. 

\begin{figure}
    \centering
    \includegraphics[width=0.7\textwidth]{images/pinhole_vs_lens}
    \caption{Difference between a pinhole camera and a lens camera. Adapted from \cite{leCameraModel2018}}
\end{figure}

\subsection{Calibration Object}

The calibration object is an object with known dimensions and features which is often used in camera calibration to 


Calibration objects can be roughly separated into 3 categories, based on the dimension of the calibration object used \footcite{zhangCameraCalibration2007}:

\begin{itemize}[leftmargin=!, itemindent=-4ex]
    \item \textbf{3D object based calibration} -- Performed by using a calibration object whose geometry is known to very high precision. Typically, the calibration object consists of 2 or 3 orthogonal planes, although a plane whose precise translation is known may also be used, which also yields 3D reference points \footcite{zhangCameraCalibration2007}. Using 3D objects is typically preferred, as it yields the highest accuracy \footcite{zhangCameraCalibration2007}, and the mathematics required is also the least. 
    \item \textbf{2D plane-based calibration} -- The most common technique is known as Zhang's method, and it requires a planar object (often a checkerboard pattern), and various pictures of this plane are taken at different orientations \footcite{zhangFlexibleNew2000}. Knowledge of the translation of the plane is not necessary. Due to its easier setup and good accuracy, it is the best choice in most situations. In fact, the most commonly used camera vision library, \texttt{OpenCV}, is geared towards this type of calibration. 
\end{itemize}

There also exists other calibration methods, notably self-calibration, which do not require calibration objects and simply rely on image correspondences. Although self-calibration does not require any calibration object, it also means that many parameters need to be estimated and optimized, and as such results in a much more complex mathematical problem \footcite{zhangCameraCalibration2007}.

For this paper, I will focus on calibration using a 3D calibration object, because the mathematics behind it is simpler, and many of the techniques used in 3D-based calibration are 
